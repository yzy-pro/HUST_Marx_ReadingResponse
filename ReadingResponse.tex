\documentclass[12pt,a4paper]{article}

% 中文支持
\usepackage{ctex}
\usepackage[UTF8]{ctex}

% 页面设置
\usepackage{geometry}
\geometry{left=2.5cm,right=2.5cm,top=2.5cm,bottom=2.5cm,headheight=15pt}

% 字体设置
\usepackage{fontspec}
\setmainfont{Times New Roman}

% 行距设置
\usepackage{setspace}
\setstretch{1.5} % 20磅行距约为1.5倍行距

% 超链接
\usepackage{hyperref}
\hypersetup{
    colorlinks=true,
    linkcolor=black,
    citecolor=black,
    urlcolor=blue
}

% 标题格式
\usepackage{titlesec}
\titleformat{\section}{\heiti\zihao{-3}\bfseries}{\thesection}{1em}{}
\titleformat{\subsection}{\heiti\zihao{4}\bfseries}{\thesubsection}{1em}{}

% 页眉页脚
\usepackage{fancyhdr}
\pagestyle{fancy}
\fancyhf{}
\fancyhead[C]{马克思主义经典著作阅读报告}
\fancyfoot[C]{\thepage}
\renewcommand{\headrulewidth}{0.4pt}
\renewcommand{\footrulewidth}{0pt}

% 首页无页眉页脚
\fancypagestyle{plain}{
    \fancyhf{}
    \renewcommand{\headrulewidth}{0pt}
    \renewcommand{\footrulewidth}{0pt}
}

\begin{document}

% 封面
\begin{titlepage}
    \centering
    \vspace*{3cm}
    
    {\heiti\zihao{1}\bfseries 马克思主义经典著作\\[0.5cm]阅读报告}
    
    \vspace{3cm}
    
    {\heiti\zihao{-2}\bfseries 《帝国主义是资本主义的最高阶段》读书报告}
    
    \vspace{4cm}
    
    \begin{spacing}{2.0}
    {\songti\zihao{-3}
    \begin{tabular}{cl}
        \textbf{院\hspace{2em}系:} & \underline{\makebox[8cm][c]{光学与电子信息学院}} \\[0.8cm]
        \textbf{班\hspace{2em}级:} & \underline{\makebox[8cm][c]{光电信息科学与工程202409班}} \\[0.8cm]
        \textbf{学\hspace{2em}号:} & \underline{\makebox[8cm][c]{U202413925}} \\[0.8cm]
        \textbf{姓\hspace{2em}名:} & \underline{\makebox[8cm][c]{陈恪瑾}} \\[0.8cm]
        \textbf{提交日期:} & \underline{\makebox[8cm][c]{2025年11月25日}} \\
    \end{tabular}
    }
    \end{spacing}
    
    \vfill
\end{titlepage}

% 设置页码从正文开始
\setcounter{page}{1}

% 正文
\section*{《帝国主义是资本主义的最高阶段》读书报告}
\addcontentsline{toc}{section}{《帝国主义是资本主义的最高阶段》读书报告}

% 三号黑体标题,正文小四号宋体,行距20磅
{\heiti\zihao{3}\textbf{一、写作背景}}

\songti\zihao{-4}
列宁的《帝国主义是资本主义的最高阶段》(以下简称《帝国主义论》)写于1916年上半年,当时正值第一次世界大战最激烈的时期。这部著作的诞生,不仅是对当时资本主义发展新阶段的理论总结,更是对第二国际机会主义路线的有力回击,为第三国际的建立奠定了坚实的理论基础。

首先,第一次世界大战的爆发是本书写作的直接历史背景。1914年爆发的第一次世界大战,是帝国主义列强为重新瓜分世界、争夺霸权而进行的一场非正义的掠夺战争。然而,战争爆发后,第二国际的各国社会党领袖(如德国的考茨基、普列汉诺夫等)却背叛了马克思主义的国际主义原则,打着“保卫祖国”的旗号,公开支持本国资产阶级政府进行战争,堕落为社会沙文主义者。这导致了第二国际的破产。面对这种严峻的形势,列宁认为必须从经济根源上揭示战争的性质,阐明帝国主义的本质,才能武装无产阶级的头脑,指明革命的方向。

其次,为了迎接社会主义革命的高潮,需要有科学的理论指导。列宁敏锐地察觉到,资本主义已经发展到了一个新的阶段——帝国主义阶段。在这个阶段,资本主义的内在矛盾空前激化,无产阶级革命的客观条件已经成熟。为了论证“帝国主义是无产阶级社会革命的前夜”这一著名论断,列宁查阅了大量的经济资料和统计数据,撰写了这部划时代的著作。

最后,这部著作也是为了在思想上和组织上与第二国际机会主义划清界限,为创建第三国际(共产国际)做准备。考茨基等人提出了“超帝国主义论”,美化帝国主义,掩盖其掠夺本质,麻痹工人阶级。列宁通过《帝国主义论》深刻批判了这些谬论,证明了帝国主义是寄生的、腐朽的、垂死的资本主义,从而为各国共产党人和革命群众提供了强大的思想武器。

{\heiti\zihao{3}\textbf{二、主要观点}}

《帝国主义论》不仅是一部经济学著作,更是一部科学社会主义的纲领性文献。列宁在书中运用历史唯物主义的观点,深刻剖析了帝国主义的经济实质和基本特征,提出了关于帝国主义的一系列科学论断。

\subsection*{(一)帝国主义的五个基本经济特征}
列宁通过对大量数据的分析,概括出了帝国主义的五个基本经济特征,这是全书的核心内容。

第一,生产和资本的集中发展到了很高的程度,以致造成了在经济生活中起决定作用的垄断组织。列宁指出,自由竞争必然引起生产集中,而生产集中发展到一定阶段就必然导致垄断。这是资本主义发展的客观规律。卡特尔、辛迪加、托拉斯等垄断组织成为全部经济生活的基础。

第二,银行资本和工业资本已经融合起来,在这个“金融资本”的基础上形成了金融寡头。银行从普通的中介人变成了万能的垄断者,它们通过信贷、持股等方式控制工业企业,形成了金融资本。金融寡头在经济上通过“参与制”实现统治,在政治上通过收买政府官员实现对国家的控制。

第三,与商品输出不同的资本输出具有了特别重要的意义。在自由资本主义阶段,占统治地位的是商品输出;而在帝国主义阶段,占统治地位的则是资本输出。过剩资本流向落后国家,因为那里资本少、地价贱、工资低、原料便宜,可以获得高额利润。资本输出成为帝国主义剥削和奴役世界的重要手段。

第四,瓜分世界的资本家国际垄断同盟已经形成。随着资本输出的增加和国外联系的扩大,各国垄断组织在世界市场上展开激烈的竞争,最终达成协议,形成国际卡特尔,在经济上瓜分世界。

第五,最大资本主义大国已把世界领土瓜分完毕。列宁指出,到19世纪末20世纪初,世界领土已经被瓜分完毕。这意味着今后的战争不再是争夺无主之地,而是重新瓜分已经瓜分了的领土。这正是帝国主义战争的根源。

\subsection*{(二)帝国主义的历史地位}
列宁在分析了上述经济特征后,进一步指出了帝国主义的历史地位,即帝国主义是垄断的、寄生的或腐朽的、垂死的资本主义。

首先,垄断是帝国主义最深厚的经济基础。垄断必然引起停滞和腐朽的趋势,例如阻碍技术进步、产生食利者阶层等。其次,帝国主义是寄生的资本主义。资本输出使得“剪息票”成为帝国主义国家的经济特征,少数高利贷国家剥削世界上大多数国家。最后,帝国主义是垂死的资本主义。垄断加剧了资本主义的基本矛盾,即生产社会化与资本主义私人占有之间的矛盾,使得经济危机更加频繁和剧烈。同时,帝国主义还激化了无产阶级与资产阶级的矛盾、宗主国与殖民地半殖民地的矛盾、帝国主义国家之间的矛盾。这些矛盾的激化,使得帝国主义成为无产阶级社会革命的前夜。

\subsection*{(三)对“超帝国主义论”的批判}
列宁在书中还着重批判了考茨基的“超帝国主义论”。考茨基认为,帝国主义的政策可以被一种新的、和平的“超帝国主义”政策所代替,即由国际金融资本联合起来共同剥削世界,从而消除战争。

列宁针锋相对地指出,这是一种极其反动的理论。首先,它掩盖了帝国主义矛盾的深刻性。帝国主义国家之间的经济发展是不平衡的,实力的变化必然导致重新瓜分世界的斗争,所谓的“联合”只是暂时的休战,战争才是不可避免的。其次,它在政治上是为机会主义辩护,企图让工人阶级放弃革命斗争,幻想通过和平手段改良资本主义。列宁的批判揭露了第二国际机会主义者的虚伪面目,捍卫了马克思主义的革命精神。

{\heiti\zihao{3}\textbf{三、所获启发}}

阅读《帝国主义论》,不仅让我对百年前的那个时代有了更清晰的认识,更让我对当今世界的政治经济格局有了深刻的洞察。结合第二国际与第三国际的历史背景,我深受启发。

\subsection*{(一)坚持马克思主义的立场、观点和方法}
列宁在撰写本书时,面对的是第二国际权威理论家的背叛和修正主义思潮的泛滥。他没有盲从权威,而是坚持用马克思主义的唯物史观去分析现实问题。他透过纷繁复杂的表面现象(如“保卫祖国”的口号),一眼看穿了战争的帝国主义本质。这启示我们,在面对复杂的国际形势和各种社会思潮时,必须保持政治定力,坚持运用马克思主义的立场、观点和方法去观察世界,透过现象看本质,才能不被各种错误观点所误导。特别是在当今信息爆炸的时代,更需要培养这种批判性思维和理论洞察力。

\subsection*{(二)深刻认识当代资本主义的新变化与本质}
虽然列宁写作本书已过去一百多年,资本主义也发生了许多新变化,如国家垄断资本主义的发展、经济全球化的深入等,但列宁揭示的帝国主义的基本特征并没有过时。
今天,跨国公司取代了当年的托拉斯,成为全球经济的主宰;金融衍生品的泛滥表明金融资本的寄生性更加严重;发达国家通过技术垄断、品牌控制和金融霸权,对发展中国家进行着更隐蔽但更残酷的剥削。这依然是列宁笔下的“资本输出”和“瓜分世界”的现代翻版。
通过阅读本书,我认识到,尽管资本主义在自我调节中获得了一定的发展空间,但其剥削本质没有变,其基本矛盾没有变。我们不能被西方国家宣扬的“民主”、“人权”等普世价值所迷惑,要看清其背后维护资本霸权的真实意图。

\subsection*{(三)坚定理想信念,反对机会主义}
回顾第二国际的破产和第三国际的建立,我深刻体会到理想信念的重要性。第二国际的领袖们在和平时期高谈阔论,一旦面临战争和危机的考验,就为了眼前的苟安和民族利己主义而背叛了革命理想。而列宁和布尔什维克党人则在最黑暗的时刻坚持真理,最终领导十月革命取得了胜利。
这对于我们今天的青年学生具有重要的现实意义。在实现中华民族伟大复兴的征程中,我们也会面临各种诱惑和挑战。我们必须像列宁那样,坚定共产主义远大理想和中国特色社会主义共同理想,坚决反对各种形式的机会主义和投降主义,在大是大非面前旗帜鲜明,在风浪考验中站稳脚跟。

{\heiti\zihao{3}\textbf{四、总结}}

《帝国主义是资本主义的最高阶段》是一座马克思主义理论的丰碑。它不仅科学地揭示了帝国主义的本质和规律,宣告了第二国际修正主义路线的破产,更为后来的俄国十月革命和全世界无产阶级革命指明了方向。

通过这次阅读,我深刻理解了为什么列宁主义是“帝国主义和无产阶级革命时代的马克思主义”。在当今世界正处于百年未有之大变局的背景下,重读这部经典著作,对于我们认清国际形势、坚定“四个自信”、投身强国伟业具有不可替代的指导意义。作为新时代的大学生,我们应当继承和发扬列宁的革命批判精神,努力学习科学文化知识,用马克思主义中国化的最新成果武装头脑,为推动构建人类命运共同体贡献自己的青春力量。

% 参考文献
\newpage
{\heiti\zihao{3}\textbf{参考文献}}

\begin{thebibliography}{99}
    \bibitem{reference1} 列宁. 帝国主义是资本主义的最高阶段[M]. 北京:人民出版社,2014.
    \bibitem{reference2} 马克思,恩格斯. 共产党宣言[M]. 北京:人民出版社,2018.
    \bibitem{reference3} 中共中央马克思恩格斯列宁斯大林著作编译局. 列宁选集(第二卷)[M]. 北京:人民出版社,2012.
    \bibitem{reference4} 陶大镛. 论列宁的《帝国主义论》及其现实意义[J]. 北京师范大学学报(社会科学版),2001(03):5-12.
    \bibitem{reference5} 顾海良. 马克思主义发展史[M]. 北京:中国人民大学出版社,2009.
\end{thebibliography}

\end{document}

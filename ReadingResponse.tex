\documentclass[12pt,a4paper]{article}

% 中文支持
\usepackage{ctex}
\usepackage[UTF8]{ctex}

% 页面设置
\usepackage{geometry}
\geometry{left=2.5cm,right=2.5cm,top=2.5cm,bottom=2.5cm,headheight=15pt}

% 字体设置
\usepackage{fontspec}
\setmainfont{Times New Roman}

% 行距设置
\usepackage{setspace}
\setstretch{1.5} % 20磅行距约为1.5倍行距

% 超链接
\usepackage{hyperref}
\hypersetup{
    colorlinks=true,
    linkcolor=black,
    citecolor=black,
    urlcolor=blue
}

% 标题格式
\usepackage{titlesec}
\titleformat{\section}{\heiti\zihao{-3}\bfseries}{\thesection}{1em}{}
\titleformat{\subsection}{\heiti\zihao{4}\bfseries}{\thesubsection}{1em}{}

% 页眉页脚
\usepackage{fancyhdr}
\pagestyle{fancy}
\fancyhf{}
\fancyhead[C]{马克思主义经典著作阅读报告}
\fancyfoot[C]{\thepage}
\renewcommand{\headrulewidth}{0.4pt}
\renewcommand{\footrulewidth}{0pt}

% 首页无页眉页脚
\fancypagestyle{plain}{
    \fancyhf{}
    \renewcommand{\headrulewidth}{0pt}
    \renewcommand{\footrulewidth}{0pt}
}

\begin{document}

% 封面
\begin{titlepage}
    \centering
    \vspace*{3cm}
    
    {\heiti\zihao{1}\bfseries 马克思主义经典著作\\[0.5cm]阅读报告}
    
    \vspace{3cm}
    
    {\heiti\zihao{-2}\bfseries 《唯物主义和经验批判主义》读书报告}
    
    \vspace{4cm}
    
    \begin{spacing}{2.0}
    {\songti\zihao{-3}
    \begin{tabular}{cl}
        \textbf{院\hspace{2em}系:} & \underline{\makebox[8cm][c]{光学与电子信息学院}} \\[0.8cm]
        \textbf{班\hspace{2em}级:} & \underline{\makebox[8cm][c]{光电信息科学与工程202409班}} \\[0.8cm]
        \textbf{学\hspace{2em}号:} & \underline{\makebox[8cm][c]{U202413925}} \\[0.8cm]
        \textbf{姓\hspace{2em}名:} & \underline{\makebox[8cm][c]{陈恪瑾}} \\[0.8cm]
        \textbf{提交日期:} & \underline{\makebox[8cm][c]{2025年11月25日}} \\
    \end{tabular}
    }
    \end{spacing}
    
    \vfill
\end{titlepage}

% 设置页码从正文开始
\setcounter{page}{1}

% 正文
\section*{《唯物主义和经验批判主义》读书报告}
\addcontentsline{toc}{section}{《唯物主义和经验批判主义》读书报告}

% 三号黑体标题,正文小四号宋体,行距20磅
{\heiti\zihao{3}\textbf{一、写作背景}}
\songti\zihao{-4}

列宁的《唯物主义和经验批判主义》写于1908年,出版于1909年。要理解这部著作的诞生,必须从欧洲自然科学的革命性变革以及俄国国内严峻的政治形势两个维度来考察。

19世纪末20世纪初,自然科学特别是物理学领域发生了一系列革命性的发现。X射线、放射性现象和电子的发现,打破了经典物理学关于原子不可分和质量不变的传统观念。面对这些新发现,一些自然科学家陷入了迷茫,惊呼“物质消失了”,导致了所谓的“物理学危机”。在这种背景下,奥地利物理学家马赫和德国哲学家阿芬那留斯创立了“经验批判主义”(即马赫主义)。他们打着“清除形而上学”的旗号,利用物理学的新发现来宣扬主观唯心主义,主张世界是“感觉要素的复合”,否认物质的客观实在性,试图用心理学取代认识论。

与此同时,这一时期的俄国正处于1905年革命失败后的斯托雷平反动时期。沙皇政府实行残酷的白色恐怖,革命运动转入低潮。在严酷的政治高压下,俄国社会民主工党内部的一些知识分子(如波格丹诺夫、巴扎罗夫、卢那察尔斯基等)在思想上发生了动摇。他们对马克思主义失去了信心,转而从马赫主义那里寻找精神寄托。他们试图用马赫主义的时髦词句来“修正”和“补充”马克思主义哲学,攻击辩证唯物主义是“陈旧的”、“神秘的”,甚至荒谬地提出了“造神说”,企图把社会主义变成一种宗教。

这种修正主义思潮在党内迅速蔓延,严重腐蚀了党的理论基础,引起了思想上的极大混乱。为了粉碎这股反动思潮,捍卫马克思主义哲学的纯洁性,列宁认为必须从哲学的高度回答自然科学革命提出的新问题,揭露马赫主义唯心主义的本质。于是,他在流亡期间查阅了大量的哲学和自然科学著作,撰写了这部战斗的唯物主义宣言书。

{\heiti\zihao{3}\textbf{二、主要观点}}

在《唯物主义和经验批判主义》中,列宁紧紧抓住唯物主义与唯心主义两条路线的斗争,通过对马赫主义“物理学唯心主义”的深刻批判,系统阐述了辩证唯物主义的物质观、认识论、真理观以及哲学的党性原则。

\subsection*{(一)马赫人物简介}
我对于马赫的初步认识,还是来源于《大学物理》第11章“波动与振动”第8节“多普勒效应”中关于“马赫锥”和“马赫数”的介绍。

除去马赫在超声速空气动力学领域的卓越成就,他在物理学的其他领域也贡献卓著,在光学(马赫带效应)、声学以及力学史研究方面有深厚造诣。最值得一提的是,马赫对牛顿绝对时空观的深刻批判,对爱因斯坦创立广义相对论产生了积极的启发作用。作为一位杰出的自然科学家,他在科学界享有崇高的声誉。

\subsection*{(二)马赫主义简介}
马赫主义,又称经验批判主义,是19世纪末20世纪初由奥地利物理学家马赫和德国哲学家阿芬那留斯创立的一种主观唯心主义哲学流派。

马赫主义的产生有着与自然科学的发展存在紧密的联系。19世纪末,物理学领域发生了一系列革命性的发现,如伦琴发现X射线、贝克勒尔发现放射性、汤姆逊发现电子等。这些新发现打破了经典物理学关于原子不可分、质量不变等传统观念,使那些把“物质”等同于“原子”的旧唯物主义者陷入了混乱。他们无法用旧有的形而上学观点解释新的科学现象,于是惊呼“原子非物质化了”、“物质消失了”,导致了所谓的“物理学危机”。

在这种背景下,唯心主义趁虚而入。马赫主义打着“清除形而上学”和“思维经济”的旗号,利用物理学的新发现来宣扬主观唯心主义。具体来说,马赫主义主张世界不是由物质构成的,而是由“感觉要素”(例如红、绿、硬、软等)构成的。他们认为“物体”不过是感觉要素的复合,科学的任务只是简练地描述这些感觉经验,而不是揭示客观实在。这种理论实质上是否认物质客观实在性的唯我论。

在俄国,波格丹诺夫、巴扎罗夫等社会民主党内的知识分子,在革命低潮时期思想动摇,试图用这种时髦的“科学哲学”来“更新”和“补充”马克思主义,宣扬“物理学唯心主义”,妄图用“经验一元论”取代辩证唯物主义。这正是列宁撰写本书进行批判的主要对象。


\subsection*{(三)马赫主义的核心观点以及具体阐释}
马赫主义的核心观点集中体现在“世界是感觉要素的复合”、“思维经济原则”和“原则同格”论上。

首先是“世界是感觉要素的复合”。马赫认为,世界万物不是由物质构成的,而是由“要素”构成的,而这些“要素”归根到底就是人的感觉。例如,当我们面对一张桌子时,我们感知到的是它的颜色、形状、硬度等。马赫主义者认为,这张桌子不过是这些感觉(要素)的集合,如果抽去了这些感觉,桌子就不复存在。这种观点将客观事物消解为主观感觉,从而否定了物质的客观实在性。

其次是“思维经济原则”。马赫主张,科学认识的目的不在于揭示客观世界的本质和规律,而在于用最少的思维消耗来“经济”地描述感觉经验。在他们看来,原子、分子等科学概念并不是客观存在的微观粒子,而只是科学家为了方便整理感觉经验而臆造出来的符号或记号。

最后是阿芬那留斯的“原则同格”。他认为“自我”(主体)和“环境”(客体)是不可分割地联系在一起的,没有无主体的客体,也没有无客体的主体。这实际上是宣扬“存在就是被感知”。按照他们的逻辑,如果没有人去感知一棵树,那么这棵树就不存在。更极端地说,在人类诞生之前的漫长地质年代里,由于没有“自我”去感知地球这个“环境”,地球也是不存在的。

\subsection*{(四)列宁批判的核心观点以及具体阐释}
针对马赫主义的唯心主义谬论,列宁在书中进行了针锋相对的批判,并系统阐述了辩证唯物主义认识论的核心观点。

第一,坚持唯物主义一元论,提出了科学的物质定义。针对马赫主义者“物质消失了”的叫嚣,列宁指出,物理学的新发现并没有推翻唯物主义,只是推翻了关于物质结构的陈旧形而上学观点。他给物质下了一个经典定义:“物质是标志客观实在的哲学范畴,这种客观实在是人通过感觉感知的,它不依赖于我们的感觉而存在,为我们的感觉所复写、摄影、反映。”这一定义指出了物质的唯一特性是“客观实在性”,从而彻底划清了唯物主义与唯心主义的界限。

第二,坚持唯物主义反映论,确立了实践在认识论中的基础地位。列宁批判了马赫主义的“原则同格”论,坚持“从物到感觉和思想”的唯物主义认识路线。他指出,感觉是客观世界的主观映象,而不是世界的“要素”。更重要的是,列宁把实践引入认识论,强调“生活、实践的观点,应该是认识论的首要的和基本的观点”。实践不仅是认识的来源和动力,更是检验真理的唯一标准,从而有力地驳斥了马赫主义的不可知论。

第三,阐明了真理的客观性以及绝对真理与相对真理的辩证关系。列宁指出,真理是客观的,因为它包含着不依赖于主体的客观内容。承认客观真理,就必须承认绝对真理。人类的认识是一个由相对真理向绝对真理无限逼近的过程。马赫主义夸大认识的相对性,陷入了相对主义和诡辩论,进而否定了客观真理的存在。

第四,鲜明地提出了哲学的党性原则。列宁一针见血地指出,哲学上唯物主义和唯心主义的斗争,归根到底表现了现代社会中敌对阶级的倾向和意识形态。马赫主义自称超越了两大派别的对立,搞“中派”哲学,实际上是掩盖其唯心主义本质,充当了宗教和反动势力的工具。列宁强调,马克思主义者在哲学战线上必须旗帜鲜明地坚持唯物主义,反对唯心主义。

\subsection*{(五)哲学的党性原则的阐释}
列宁在本书的最后部分,深刻阐释了哲学的党性原则。他指出,哲学上的党性就是唯物主义和唯心主义的对立和斗争。这种斗争归根到底表现了现代社会中敌对阶级的倾向和意识形态。

首先,没有超阶级的、非党性的哲学。任何哲学体系,无论它打着多么新奇的旗号,最终都要回答“物质和意识谁是第一性”的问题,从而归属于唯物主义或唯心主义两大阵营。马赫主义者自称超越了两大派别的对立,搞“中派”哲学,实际上是掩盖其唯心主义的本质,充当了宗教和反动势力的工具。

其次,坚持哲学的党性原则,就是要求马克思主义者在哲学战线上必须旗帜鲜明。不能搞折衷主义,不能对唯心主义妥协退让。列宁强调,马克思主义哲学是无产阶级的世界观,它公开申明为无产阶级利益服务。只有坚持辩证唯物主义,才能彻底战胜各种形式的唯心主义和不可知论,为无产阶级革命提供科学的指导思想。这一原则对于我们今天识别各种错误思潮、保持政治定力具有极强的现实意义。

{\heiti\zihao{3}\textbf{三、所获启发}}

阅读《唯物主义和经验批判主义》,不仅让我对百年前的哲学论战有了清晰的了解,更让我深刻体会到了马克思主义哲学的强大生命力和现实指导意义。

\subsection*{(一)坚持物质第一性}
坚持物质第一性,是辩证唯物主义的基石。列宁在书中通过对“物理学唯心主义”的批判,再次确立了物质不依赖于意识而存在的客观地位。这对我们今天的学习和生活具有重要的指导意义。

首先,它要求我们必须尊重客观事实。在面对复杂的社会现象或科学问题时,我们不能从主观愿望或抽象的定义出发,而必须从客观存在的实际情况出发。比如在进行科研实验时,必须忠实于实验数据,不能为了凑结论而篡改数据;在分析社会热点时,要深入调查研究,探寻其背后的物质利益根源,而不是被表面的舆论喧嚣所迷惑。

其次,坚持物质第一性意味着要反对一切形式的主观主义和唯意志论。我们不能幻想通过单纯的“意念”或“精神力量”去改变现实,而必须通过实实在在的物质实践活动去改造世界。只有承认世界的客观性,我们的主观能动性才能在尊重客观规律的基础上得到正确的发挥。

\subsection*{(二)把握认识论规律}
把握认识论规律,核心在于深刻理解绝对真理与相对真理的辩证关系。列宁指出,绝对真理是由相对真理构成的,相对真理是通向绝对真理的阶梯。这一观点对于我们今天的学习和科研具有极强的指导价值。

首先,我们要树立科学的真理观,反对绝对主义和相对主义两种极端。在学习过程中,我们既不能把现有的理论当作僵死的教条,认为它们是终极真理而拒绝发展;也不能因为科学理论会随着时代更新而否定其客观真理的成分,陷入不可知论的泥潭。例如,牛顿力学虽然被相对论超越,但它在宏观低速领域的真理性依然存在。我们应当认识到,每一个科学结论都是人类认识长河中的一个里程碑,它包含着绝对真理的颗粒,但又受到历史条件的限制。

其次,我们要坚持实践是检验真理的唯一标准。认识不是一次完成的,而是一个“实践、认识、再实践、再认识”的无限循环过程。作为新时代的大学生,我们在掌握书本知识的同时,更要注重社会实践和科学实验。只有在实践中,我们才能检验知识的真伪,发现新的问题,从而推动认识的不断深化。这种勇于实践、敢于创新的精神,正是辩证唯物主义认识论赋予我们的宝贵财富。

\subsection*{(三)正确看待自然科学变革中的“危机”}
列宁在书中深刻分析了当时的“物理学危机”,一针见血地指出这其实是“形而上学唯物主义的危机”,而不是“唯物主义的崩溃”。这一论断犹如灯塔,对于我们今天面对日新月异的科技发展具有极强的现实指引意义。

首先,我们要学会用辩证唯物主义的观点去审视现代科学技术的新成果。当今世界,量子力学、人工智能、生物基因等前沿技术飞速发展,微观粒子的波粒二象性、量子纠缠等现象极大地挑战了人们的传统认知。就像当年的马赫主义者利用电子的发现宣扬唯心主义一样,今天也有人试图利用量子力学的不确定性来否定因果律,或者利用人工智能的发展来宣扬“意识仅仅是算法”的机械唯物论,甚至鼓吹“世界是虚拟的”等不可知论观点。列宁的教导告诉我们,无论科学如何发展,物质的客观实在性这一根本属性不会改变。科学的新发现只会不断丰富我们对物质存在形式的认识,证明辩证唯物主义的正确性,而绝不会推翻唯物主义的基石。

其次,我们要深刻认识到自然科学与哲学的紧密联系。列宁曾告诫:“自然科学家如果不采取辩证唯物主义的立场,就无法在资产阶级思想的侵袭和资产阶级世界观的复辟面前坚持住。”作为光电信息科学与工程专业的学生,我们在深入钻研专业知识、探索光子与电子奥秘的同时,必须自觉提高哲学素养。只有掌握了辩证唯物主义这一科学的世界观和方法论,我们才能在探索自然奥秘的过程中不迷失方向,透过复杂的现象把握事物的本质,避免滑向唯心主义或形而上学的深渊,真正成为一名既有科学精神又有哲学头脑的新时代科技人才。

% % 参考文献
% \newpage
% {\heiti\zihao{3}\textbf{参考文献}}

% \begin{thebibliography}{99}
%     \bibitem{reference1} 列宁. 唯物主义和经验批判主义[M]. 北京:人民出版社,2014.
%     \bibitem{reference2} 中共中央马克思恩格斯列宁斯大林著作编译局. 列宁选集(第二卷)[M]. 北京:人民出版社,2012.
%     \bibitem{reference3} 曹浩瀚. 列宁《唯物主义和经验批判主义》的当代价值[J]. 马克思主义研究,2019(05):25-34.
%     \bibitem{reference4} 孙正聿. 辩证唯物主义的“世界图景”——重读《唯物主义和经验批判主义》[J]. 哲学研究,2009(04):3-10.
%     \bibitem{reference5} 顾海良. 马克思主义发展史[M]. 北京:中国人民大学出版社,2009.
% \end{thebibliography}

\end{document}

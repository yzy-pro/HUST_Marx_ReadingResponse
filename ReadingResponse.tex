\documentclass[12pt,a4paper]{article}

% 中文支持
\usepackage{ctex}
\usepackage[UTF8]{ctex}

% 页面设置
\usepackage{geometry}
\geometry{left=2.5cm,right=2.5cm,top=2.5cm,bottom=2.5cm,headheight=15pt}

% 字体设置
\usepackage{fontspec}
\setmainfont{Times New Roman}

% 行距设置
\usepackage{setspace}
\setstretch{1.5} % 20磅行距约为1.5倍行距

% 超链接
\usepackage{hyperref}
\hypersetup{
    colorlinks=true,
    linkcolor=black,
    citecolor=black,
    urlcolor=blue
}

% 标题格式
\usepackage{titlesec}
\titleformat{\section}{\heiti\zihao{-3}\bfseries}{\thesection}{1em}{}
\titleformat{\subsection}{\heiti\zihao{4}\bfseries}{\thesubsection}{1em}{}

% 页眉页脚
\usepackage{fancyhdr}
\pagestyle{fancy}
\fancyhf{}
\fancyhead[C]{马克思主义经典著作阅读报告}
\fancyfoot[C]{\thepage}
\renewcommand{\headrulewidth}{0.4pt}
\renewcommand{\footrulewidth}{0pt}

% 首页无页眉页脚
\fancypagestyle{plain}{
    \fancyhf{}
    \renewcommand{\headrulewidth}{0pt}
    \renewcommand{\footrulewidth}{0pt}
}

\begin{document}

% 封面
\begin{titlepage}
    \centering
    \vspace*{3cm}
    
    {\heiti\zihao{1}\bfseries 马克思主义经典著作\\[0.5cm]阅读报告}
    
    \vspace{3cm}
    
    {\heiti\zihao{-2}\bfseries 《经典著作名称》读书报告}
    
    \vspace{4cm}
    
    \begin{spacing}{2.0}
    {\songti\zihao{-3}
    \begin{tabular}{cl}
        \textbf{院\hspace{2em}系:} & \underline{\makebox[8cm][c]{光学与电子信息学院}} \\[0.8cm]
        \textbf{班\hspace{2em}级:} & \underline{\makebox[8cm][c]{光电信息科学与工程202409班}} \\[0.8cm]
        \textbf{学\hspace{2em}号:} & \underline{\makebox[8cm][c]{U202413925}} \\[0.8cm]
        \textbf{姓\hspace{2em}名:} & \underline{\makebox[8cm][c]{陈恪瑾}} \\[0.8cm]
        \textbf{提交日期:} & \underline{\makebox[8cm][c]{2025年11月25日}} \\
    \end{tabular}
    }
    \end{spacing}
    
    \vfill
\end{titlepage}

% 设置页码从正文开始
\setcounter{page}{1}

% 正文
\section*{《经典著作名称》读书报告}
\addcontentsline{toc}{section}{《经典著作名称》读书报告}

% 三号黑体标题,正文小四号宋体,行距20磅
{\heiti\zihao{3}\textbf{一、写作背景}}

\songti\zihao{-4}
在这里简要介绍经典著作的写作背景,包括历史背景、社会背景、作者的写作动机等。这部分内容可以略写。

例如:《共产党宣言》写于1847年底至1848年初,当时欧洲正处于资本主义迅速发展时期,无产阶级与资产阶级的矛盾日益尖锐。马克思和恩格斯受共产主义者同盟委托,起草了这份纲领性文件。

{\heiti\zihao{3}\textbf{二、主要观点}}

在这里系统阐述经典著作的主要观点和核心思想,这是报告的重点部分。建议分点论述,条理清晰。

\subsection*{(一)第一个主要观点}
详细阐述第一个主要观点的内容、论证过程和理论意义。可以引用原文,并进行分析和评论。

\subsection*{(二)第二个主要观点}
详细阐述第二个主要观点的内容、论证过程和理论意义。

\subsection*{(三)第三个主要观点}
详细阐述第三个主要观点的内容、论证过程和理论意义。

{\heiti\zihao{3}\textbf{三、所获启发}}

在这里谈谈阅读这部经典著作后的收获、启发和思考,可以联系实际,谈谈对当代社会的启示意义。

\subsection*{(一)理论层面的启发}
从理论角度谈谈自己的收获和思考。

\subsection*{(二)实践层面的启发}
从实践角度谈谈如何将经典理论应用于分析和解决现实问题。

\subsection*{(三)个人成长的启发}
谈谈阅读经典著作对个人思想认识和价值观的影响。

{\heiti\zihao{3}\textbf{四、总结}}

对整个阅读过程和收获进行总结,可以谈谈自己对马克思主义经典著作的认识,以及今后的学习计划。

注意:正文字数不少于2500字,重复率低于20\%。

% 参考文献
\newpage
{\heiti\zihao{3}\textbf{参考文献}}

\begin{thebibliography}{99}
    \bibitem{reference1} 马克思,恩格斯. 共产党宣言[M]. 北京:人民出版社,2018.
    
    \bibitem{reference2} 作者姓名. 文章题目[J]. 期刊名称,年份,卷号(期号):起止页码.
    
    \bibitem{reference3} 作者姓名. 书名[M]. 出版地:出版社,出版年份.
    
    \bibitem{reference4} 作者姓名. 学位论文题目[D]. 城市:大学名称,年份.
    
    \bibitem{reference5} 作者姓名. 论文题目[C]//编者. 论文集名称. 出版地:出版社,年份:起止页码.
\end{thebibliography}

\end{document}

\documentclass[12pt,a4paper]{article}

% 中文支持
\usepackage{ctex}
\usepackage[UTF8]{ctex}

% 页面设置
\usepackage{geometry}
\geometry{left=2.5cm,right=2.5cm,top=2.5cm,bottom=2.5cm,headheight=15pt}

% 字体设置
\usepackage{fontspec}
\setmainfont{Times New Roman}

% 行距设置
\usepackage{setspace}
\setstretch{1.5} % 20磅行距约为1.5倍行距

% 超链接
\usepackage{hyperref}
\hypersetup{
    colorlinks=true,
    linkcolor=black,
    citecolor=black,
    urlcolor=blue
}

% 标题格式
\usepackage{titlesec}
\titleformat{\section}{\heiti\zihao{-3}\bfseries}{\thesection}{1em}{}
\titleformat{\subsection}{\heiti\zihao{4}\bfseries}{\thesubsection}{1em}{}

% 页眉页脚
\usepackage{fancyhdr}
\pagestyle{fancy}
\fancyhf{}
\fancyhead[C]{马克思主义经典著作阅读报告}
\fancyfoot[C]{\thepage}
\renewcommand{\headrulewidth}{0.4pt}
\renewcommand{\footrulewidth}{0pt}

% 首页无页眉页脚
\fancypagestyle{plain}{
    \fancyhf{}
    \renewcommand{\headrulewidth}{0pt}
    \renewcommand{\footrulewidth}{0pt}
}

\begin{document}

% 封面
\begin{titlepage}
    \centering
    \vspace*{3cm}
    
    {\heiti\zihao{1}\bfseries 马克思主义经典著作\\[0.5cm]阅读报告}
    
    \vspace{3cm}
    
    {\heiti\zihao{-2}\bfseries 《唯物主义和经验批判主义》读书报告}
    
    \vspace{4cm}
    
    \begin{spacing}{2.0}
    {\songti\zihao{-3}
    \begin{tabular}{cl}
        \textbf{院\hspace{2em}系:} & \underline{\makebox[8cm][c]{光学与电子信息学院}} \\[0.8cm]
        \textbf{班\hspace{2em}级:} & \underline{\makebox[8cm][c]{光电信息科学与工程202409班}} \\[0.8cm]
        \textbf{学\hspace{2em}号:} & \underline{\makebox[8cm][c]{U202413925}} \\[0.8cm]
        \textbf{姓\hspace{2em}名:} & \underline{\makebox[8cm][c]{陈恪瑾}} \\[0.8cm]
        \textbf{提交日期:} & \underline{\makebox[8cm][c]{2025年11月25日}} \\
    \end{tabular}
    }
    \end{spacing}
    
    \vfill
\end{titlepage}

% 设置页码从正文开始
\setcounter{page}{1}

% 正文
\section*{《唯物主义和经验批判主义》读书报告}
\addcontentsline{toc}{section}{《唯物主义和经验批判主义》读书报告}

% 三号黑体标题,正文小四号宋体,行距20磅
{\heiti\zihao{3}\textbf{一、写作背景}}

\songti\zihao{-4}
列宁的《唯物主义和经验批判主义》写于1908年,出版于1909年。当时俄国正处于1905年革命失败后的斯托雷平反动时期。沙皇政府在政治上进行残酷镇压的同时,思想界也掀起了一股攻击马克思主义的逆流。

在俄国社会民主工党内部,一些知识分子(如波格丹诺夫、巴扎罗夫等)受到奥地利物理学家马赫和阿芬那留斯创立的“经验批判主义”哲学的影响,试图用这种时髦的唯心主义哲学来“修正”和“补充”马克思主义哲学。他们攻击辩证唯物主义是“陈旧的”、“神秘的”,宣扬“物理学唯心主义”,否认物质的客观实在性,主张世界是“感觉要素的复合”。这种修正主义思潮严重腐蚀了党的理论基础,引起了思想上的混乱。

为了粉碎这股修正主义思潮,捍卫马克思主义哲学的纯洁性,划清唯物主义与唯心主义的界限,统一全党的思想,迎接新的革命高潮,列宁在流亡日内瓦和伦敦期间,查阅了大量的哲学和自然科学著作,撰写了这部哲学巨著。它不仅是对马赫主义的全面批判,也是对辩证唯物主义认识论的深刻阐述。

{\heiti\zihao{3}\textbf{二、主要观点}}

《唯物主义和经验批判主义》是一部论战性的哲学著作,列宁在书中紧紧抓住唯物主义和唯心主义这两条哲学路线的斗争,系统阐述了辩证唯物主义的认识论。

\subsection*{(一)物质的定义与唯物主义基石}
针对马赫主义者否认物质客观实在性的谬论,列宁给物质下了一个经典的定义:“物质是标志客观实在的哲学范畴,这种客观实在是人通过感觉感知的,它不依赖于我们的感觉而存在,为我们的感觉所复写、摄影、反映。”

这个定义包含三层含义:第一,物质的唯一特性是“客观实在性”,这是相对于意识而言的;第二,物质是可知的,人可以通过感觉感知它;第三,物质第一性,意识第二性,意识是物质的反映。列宁通过这个定义,彻底划清了唯物主义与唯心主义、不可知论的界限,有力地批判了马赫主义将“要素”(感觉)作为世界本原的主观唯心主义观点。

\subsection*{(二)认识论:反映论与实践观}
列宁坚持了从物到感觉和思想的唯物主义认识路线,批判了从思想和感觉到物的唯心主义路线。他指出,认识是客观世界在人脑中的反映,但这种反映不是僵死的、照镜子式的,而是一个充满矛盾的辩证过程。

列宁特别强调了实践在认识论中的地位,指出“生活、实践的观点,应该是认识论的首要的和基本的观点”。实践不仅是认识的来源和动力,更是检验真理的唯一标准。他批评马赫主义把实践排除在认识论之外,把认识局限在个人的感觉经验之中,从而陷入了唯我论的泥潭。

\subsection*{(三)真理的客观性与辩证法}
列宁论述了真理的客观性,指出真理的内容是客观的,不以人的意志为转移。承认客观真理,就是坚持唯物主义。

同时,他阐明了绝对真理和相对真理的辩证关系:人类的认识是无限发展的,每一个科学真理都是对绝对真理的相对接近,无数相对真理的总和构成绝对真理。绝对真理寓于相对真理之中,通过相对真理表现出来。这既反对了认为真理一成不变的教条主义(绝对主义),也反对了否认客观真理存在的相对主义。列宁指出,马赫主义正是利用相对主义来否定客观真理,进而否定唯物主义的。

\subsection*{(四)哲学的党性原则}
列宁鲜明地提出了哲学的党性原则。他指出,哲学上唯物主义和唯心主义的斗争,归根到底表现了现代社会中敌对阶级的倾向和意识形态。没有超阶级的、非党性的哲学。

马赫主义者自称超越了唯物主义和唯心主义的对立,搞“中派”哲学,实际上是掩盖其唯心主义的本质,充当了宗教和反动势力的工具。列宁强调,马克思主义者在哲学战线上必须旗帜鲜明地坚持唯物主义,反对唯心主义,不能搞折衷主义。

{\heiti\zihao{3}\textbf{三、所获启发}}

阅读《唯物主义和经验批判主义》,不仅让我对百年前的哲学论战有了清晰的了解,更让我深刻体会到了马克思主义哲学的强大生命力和现实指导意义。

\subsection*{(一)坚持实事求是的思想路线}
列宁在书中反复强调物质的客观实在性,这启示我们在任何时候都要坚持唯物主义的根本立场。在当今信息爆炸、各种思潮泛滥的时代,我们更要坚持一切从实际出发,不唯书、不唯上、只唯实。

无论是在学习还是工作中,我们都不能凭主观臆断办事,而要深入实际调查研究,尊重客观规律。例如,在进行科学研究时,必须以客观实验数据为依据,不能为了迎合某种理论而篡改数据;在分析社会问题时,要从客观存在的社会物质生活条件出发,而不是从抽象的观念出发。

\subsection*{(二)树立科学的真理观}
列宁关于绝对真理和相对真理辩证关系的论述,对于我们树立科学的真理观具有重要的指导意义。我们既要承认真理的客观性,反对虚无主义;又要认识到真理的发展性,反对教条主义。

在求知过程中,我们要明白,人类的认识是永无止境的。我们现有的知识只是相对真理,需要随着实践的发展而不断深化和完善。因此,我们要保持开放的心态,勇于探索、不断创新,敢于突破前人的局限,但同时也要尊重前人积累的科学成果,不能搞历史虚无主义。

\subsection*{(三)增强理论辨别力,保持政治清醒}
列宁当年面对的是披着“最新自然科学”外衣的马赫主义,今天我们也面临着各种形式的新唯心主义思潮。它们往往打着“后现代”、“解构主义”或某些前沿科技的旗号,否定客观真理,宣扬相对主义和不可知论。

通过阅读本书,我认识到哲学斗争往往反映了背后的阶级利益和政治倾向。我们需要学习列宁的战斗精神,善于运用马克思主义的显微镜和望远镜,透过现象看本质,识别各种错误思潮的实质。在复杂的意识形态斗争中,我们要保持清醒的头脑,坚定马克思主义信仰,自觉抵制各种唯心主义和形而上学的侵蚀。

{\heiti\zihao{3}\textbf{四、总结}}

《唯物主义和经验批判主义》不仅是对俄国党内修正主义的清算,也是对辩证唯物主义认识论的全面发展。它被誉为“马克思主义哲学的百科全书”。

通过这次阅读,我深刻体会到了哲学不仅是书斋里的学问,更是认识世界和改造世界的锐利武器。列宁那种坚持真理、毫不妥协的革命精神,以及严谨治学、逻辑严密的理论风范,都深深打动了我。在今后的学习和生活中,我将继续深入学习马克思主义经典著作,努力掌握辩证唯物主义的世界观和方法论,不断提高自己的理论素养和实践能力,做一名坚定的马克思主义者。

% 参考文献
\newpage
{\heiti\zihao{3}\textbf{参考文献}}

\begin{thebibliography}{99}
    \bibitem{reference1} 列宁. 唯物主义和经验批判主义[M]. 北京:人民出版社,2014.
    \bibitem{reference2} 中共中央马克思恩格斯列宁斯大林著作编译局. 列宁选集(第二卷)[M]. 北京:人民出版社,2012.
    \bibitem{reference3} 曹浩瀚. 列宁《唯物主义和经验批判主义》的当代价值[J]. 马克思主义研究,2019(05):25-34.
    \bibitem{reference4} 孙正聿. 辩证唯物主义的“世界图景”——重读《唯物主义和经验批判主义》[J]. 哲学研究,2009(04):3-10.
    \bibitem{reference5} 顾海良. 马克思主义发展史[M]. 北京:中国人民大学出版社,2009.
\end{thebibliography}

\end{document}

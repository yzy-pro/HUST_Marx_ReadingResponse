\documentclass[12pt,a4paper]{article}

% 中文支持
\usepackage{ctex}
\usepackage[UTF8]{ctex}

% 页面设置
\usepackage{geometry}
\geometry{left=2.5cm,right=2.5cm,top=2.5cm,bottom=2.5cm,headheight=15pt}

% 字体设置
\usepackage{fontspec}
\setmainfont{Times New Roman}

% 行距设置
\usepackage{setspace}
\setstretch{1.5} % 20磅行距约为1.5倍行距

% 超链接
\usepackage{hyperref}
\hypersetup{
    colorlinks=true,
    linkcolor=black,
    citecolor=black,
    urlcolor=blue
}

% 标题格式
\usepackage{titlesec}
\titleformat{\section}{\heiti\zihao{-3}\bfseries}{\thesection}{1em}{}
\titleformat{\subsection}{\heiti\zihao{4}\bfseries}{\thesubsection}{1em}{}

% 页眉页脚
\usepackage{fancyhdr}
\pagestyle{fancy}
\fancyhf{}
\fancyhead[C]{马克思主义经典著作阅读报告}
\fancyfoot[C]{\thepage}
\renewcommand{\headrulewidth}{0.4pt}
\renewcommand{\footrulewidth}{0pt}

% 首页无页眉页脚
\fancypagestyle{plain}{
    \fancyhf{}
    \renewcommand{\headrulewidth}{0pt}
    \renewcommand{\footrulewidth}{0pt}
}

\begin{document}

% 封面
\begin{titlepage}
    \centering
    \vspace*{3cm}
    
    {\heiti\zihao{1}\bfseries 马克思主义经典著作\\[0.5cm]阅读报告}
    
    \vspace{3cm}
    
    {\heiti\zihao{-2}\bfseries 《唯物主义和经验批判主义》读书报告}
    
    \vspace{4cm}
    
    \begin{spacing}{2.0}
    {\songti\zihao{-3}
    \begin{tabular}{cl}
        \textbf{院\hspace{2em}系:} & \underline{\makebox[8cm][c]{光学与电子信息学院}} \\[0.8cm]
        \textbf{班\hspace{2em}级:} & \underline{\makebox[8cm][c]{光电信息科学与工程202409班}} \\[0.8cm]
        \textbf{学\hspace{2em}号:} & \underline{\makebox[8cm][c]{U202413925}} \\[0.8cm]
        \textbf{姓\hspace{2em}名:} & \underline{\makebox[8cm][c]{陈恪瑾}} \\[0.8cm]
        \textbf{提交日期:} & \underline{\makebox[8cm][c]{2025年11月25日}} \\
    \end{tabular}
    }
    \end{spacing}
    
    \vfill
\end{titlepage}

% 设置页码从正文开始
\setcounter{page}{1}

% 正文
\section*{《唯物主义和经验批判主义》读书报告}
\addcontentsline{toc}{section}{《唯物主义和经验批判主义》读书报告}

% 三号黑体标题,正文小四号宋体,行距20磅
{\heiti\zihao{3}\textbf{一、写作背景}\footnote{本部分内容参考了百度百科,并使用了AI工具辅助撰写。}}
\songti\zihao{-4}

列宁的《唯物主义和经验批判主义》写于1908年,出版于1909年。要理解这部著作的诞生,必须从欧洲自然科学的革命性变革以及俄国国内严峻的政治形势两个维度来考察。

19世纪末20世纪初,自然科学特别是物理学领域发生了一系列革命性的发现。X射线、放射性现象和电子的发现,打破了经典物理学关于原子不可分和质量不变的传统观念。面对这些新发现,一些自然科学家陷入了迷茫,惊呼“物质消失了”,导致了所谓的“物理学危机”。在这种背景下,奥地利物理学家马赫和德国哲学家阿芬那留斯创立了“经验批判主义”(即马赫主义)。他们打着“清除形而上学”的旗号,利用物理学的新发现来宣扬主观唯心主义,主张世界是“感觉要素的复合”,否认物质的客观实在性,试图用心理学取代认识论。

与此同时,这一时期的俄国正处于1905年革命失败后的斯托雷平反动时期。沙皇政府实行残酷的白色恐怖,革命运动转入低潮。在严酷的政治高压下,俄国社会民主工党内部的一些知识分子(如波格丹诺夫、巴扎罗夫、卢那察尔斯基等)在思想上发生了动摇。他们对马克思主义失去了信心,转而从马赫主义那里寻找精神寄托。他们试图用马赫主义的时髦词句来“修正”和“补充”马克思主义哲学,攻击辩证唯物主义是“陈旧的”、“神秘的”,甚至荒谬地提出了“造神说”,企图把社会主义变成一种宗教。

这种修正主义思潮在党内迅速蔓延,严重腐蚀了党的理论基础,引起了思想上的极大混乱。为了粉碎这股反动思潮,捍卫马克思主义哲学的纯洁性,列宁认为必须从哲学的高度回答自然科学革命提出的新问题,揭露马赫主义唯心主义的本质。于是,他在流亡期间查阅了大量的哲学和自然科学著作,撰写了这部战斗的唯物主义宣言书。

{\heiti\zihao{3}\textbf{二、主要观点}}

在《唯物主义和经验批判主义》中,列宁紧紧抓住唯物主义与唯心主义两条路线的斗争,通过对马赫主义“物理学唯心主义”的深刻批判,系统阐述了辩证唯物主义的物质观、认识论、真理观以及哲学的党性原则。\footnote{这个段落是完成后文分点论述后使用AI总结的}

\subsection*{(一)马赫人物简介}
对于马赫的初步认识,我还是来源于《大学物理》\footnote{项林川. 大学物理[M]. 北京:高等教育出版社, 2021.} 第11章“波动与振动”第8节“多普勒效应”中关于“马赫数”\footnote{马赫数被定义为物体速度与当地声速的比值,用以描述物体在空气中运动时产生的冲击波现象}和“马赫锥”\footnote{马赫锥是指当物体速度超过声速时,所形成的锥形冲击波结构。}的介绍。当时我并未意识到我会在后续马克思主义基本原理的学习过程中再次遇到他。

马赫除去在超声速空气动力学领域的卓越成就,他在物理学的其他领域也贡献卓著,在光学(马赫带效应)、声学以及力学史研究方面有深厚造诣。最值得一提的是,马赫对牛顿绝对时空观的深刻批判,对爱因斯坦创立广义相对论产生了积极的启发作用。\footnote{http://mp.ihep.ac.cn/article/pdf/preview/9014}作为一位杰出的自然科学家,他在科学界毫无疑问地享有崇高的声誉。

\subsection*{(二)马赫主义简介}
马赫主义,又称经验批判主义,是19世纪末20世纪初由奥地利物理学家马赫和德国哲学家阿芬那留斯创立的一种主观唯心主义哲学流派。

马赫主义的产生有着与自然科学的发展存在紧密的联系。19世纪末,物理学领域发生了一系列革命性的发现,如伦琴发现X射线、贝克勒尔发现放射性、汤姆逊发现电子等。这些新发现打破了经典物理学关于原子不可分、质量不变等传统观念,使那些把“物质”等同于“原子”的旧唯物主义者陷入了混乱。他们无法用旧有的形而上学观点解释新的科学现象,于是惊呼“原子非物质化了”、“物质消失了”,导致了所谓的“物理学危机”。

在这种背景下,唯心主义趁虚而入。马赫主义打着“清除形而上学”和“思维经济”的旗号,利用物理学的新发现来宣扬主观唯心主义。具体来说,马赫主义主张世界不是由物质构成的,而是由“感觉要素”(例如红、绿、硬、软等)构成的。他们认为“物体”不过是感觉要素的复合,科学的任务只是简练地描述这些感觉经验,而不是揭示客观实在。这种理论实质上是否认物质客观实在性的唯我论。

在俄国,波格丹诺夫、巴扎罗夫等社会民主党内的知识分子,在革命低潮时期思想动摇,试图用这种时髦的“科学哲学”来“更新”和“补充”马克思主义,宣扬“物理学唯心主义”,妄图用“经验一元论”取代辩证唯物主义。这正是列宁撰写本书进行批判的主要对象。


\subsection*{(三)马赫主义的核心观点以及具体阐释}\footnote{此部分的举例说明参考了诸多网络博客对马赫主义的生动解释,由于参考资料较多,且权威性较低,无法一一列出,特此说明。}
马赫主义的核心观点集中体现在“世界的本质是感觉要素的复合”、“思维经济”和“原则同格”三个核心论点上。

首先是“世界的本质是感觉要素的复合”。马赫认为,世界万物不是由物质构成的,而是由“要素”构成的,不同于化学元素或分子原子等,这些“要素”归根到底就是人的感觉。例如,当我们面对一张桌子时,我们能感知到它的颜色、形状、硬度等,但我们只会认为这只是桌子的物理属性,而不是构成桌子的本质。而马赫主义者认为,这张桌子不过是这些感觉(要素)的集合,如果抽去了这些感觉,桌子就不复存在。这种观点将客观事物消解为主观感觉,否定了物质的客观实在性。

其次是“思维经济”。马赫认为,科学认识的目的不在于揭示客观世界的本质和规律,而在于用最少的思维消耗来“经济”地描述感觉经验。在他们看来,原子、分子等科学概念并不是客观存在的微观粒子,只是因为这些概念的产生可以减少人们理解认识世界的思维量,于是科学家为了方便整理感觉经验而臆造出来这些符号或记号。这些概念并不反映客观实在,而只是为了更经济的思维而诞生的工具。如同费马原理\footnote{费马原理是几何光学的基本原理,其核心是:光从一点传播到另一点的实际路径,是使光程取极值的路径。}中光总是沿极值路径传播,马赫主义认为科学概念的阐释总是为了达到思维最“经济”目的。

最后是阿芬那留斯的“原则同格”。他认为“自我”(主体)和“环境”(客体)是不可分割地联系在一起的,没有无主体的客体,也没有无客体的主体。这实际上是宣扬“存在就是被感知”。按照这样的逻辑,如果我没有去感知一棵树,那么这棵树就不存在。更极端地说,在人类诞生之前的漫长地质年代里,由于没有“自我”去感知地球这个“环境”,地球也是不存在的。

\subsection*{(四)列宁批判的核心观点以及具体阐释}
针对马赫主义的唯心主义谬论,列宁在书中进行了针锋相对的批判,并系统阐述了辩证唯物主义认识论的核心观点。

第一,列宁坚持唯物主义一元论,并提出了科学的物质定义。针对马赫主义者“物质消失了”的论调,列宁指出,物理学的新发现并没有推翻唯物主义,只是推翻了关于物质结构的陈旧形而上学观点。在书中,列宁给物质下了一个定义:“物质是标志客观实在的哲学范畴,这种客观实在是人通过感觉感知的,它不依赖于我们的感觉而存在,为我们的感觉所复写、摄影、反映。”\footnote{列宁. 唯物主义和经验批判主义. 列宁选集(第18卷). 第二章第四节.}这一定义指出了物质的唯一特性是“客观实在性”,从而彻底划清了唯物主义与唯心主义的界限。

第二,坚持唯物主义反映论,确立了实践在认识论中的基础地位。列宁批判了马赫主义的“原则同格”,坚持“从物到感觉和思想”\footnote{列宁. 唯物主义和经验批判主义. 列宁选集(第18卷). 第一章第一节.}的唯物主义认识路线。他指出,感觉是客观世界的主观映象,而不是世界的“要素”。更重要的是,列宁把实践引入认识论,强调“生活、实践的观点,应该是认识论的首要的和基本的观点”\footnote{列宁. 唯物主义和经验批判主义. 列宁选集(第18卷). 第二章第六节.}。实践不仅是认识的来源和动力,更是检验真理的唯一标准,从而有力地驳斥了马赫主义的不可知论。

第三,阐明了真理的客观性以及绝对真理与相对真理的辩证关系。列宁指出,真理是客观的,因为它包含着不依赖于主体的客观内容。承认客观真理,就必须承认绝对真理。人类的认识是一个由相对真理向绝对真理无限逼近的过程。马赫主义夸大认识的相对性,陷入了相对主义和诡辩论,进而否定了客观真理的存在。

第四,鲜明地提出了哲学的党性原则\footnote{在阅读的过程中我对“哲学的党性原则”不甚理解,此内容将在下一部分详细讨论}。列宁指出,哲学上唯物主义和唯心主义的斗争,归根到底表现了现代社会中敌对阶级的倾向和意识形态。马赫主义自认为提出了超越了两大派别的对立,介于二者之间的哲学,但其实际上不过是掩盖其唯心主义本质的唯心主义,披着超越唯心主义外衣的唯心主义。在书中,列宁强调,马克思主义者在哲学战线上必须坚定地坚持唯物主义,反对唯心主义。

\subsection*{(五)哲学的党性原则的阐释}
列宁在本书的最后部分,深刻阐释了哲学的党性原则。他指出,哲学上的党性就是唯物主义和唯心主义的对立和斗争。这种斗争归根到底表现了现代社会中敌对阶级的倾向和意识形态。

首先,没有超阶级的、非党性的哲学。任何哲学体系,无论它打着多么新奇的旗号,最终都要回答“物质和意识谁是第一性”的问题,从而归属于唯物主义或唯心主义两大阵营。马赫主义者自称超越了两大派别的对立,搞“中派”哲学,实际上是掩盖其唯心主义的本质,充当了宗教和反动势力的工具。

其次,坚持哲学的党性原则,就是要求马克思主义者在哲学战线上必须旗帜鲜明。不能搞折衷主义,不能对唯心主义妥协退让。列宁强调,马克思主义哲学是无产阶级的世界观,它公开申明为无产阶级利益服务。只有坚持辩证唯物主义,才能彻底战胜各种形式的唯心主义和不可知论,为无产阶级革命提供科学的指导思想。这一原则对于我们今天识别各种错误思潮、保持政治定力具有极强的现实意义。

{\heiti\zihao{3}\textbf{三、所获启发}}

阅读《唯物主义和经验批判主义》,让我深刻体会到了马克思主义哲学的强大生命力和现实指导意义。

\subsection*{(一)坚持物质第一性}
坚持物质第一性,是辩证唯物主义的基石。习近平总书记也曾指出:“世界物质统一性原理是辩证唯物主义最基本、最核心的观点,是马克思主义哲学的基石。\footnote{福建省市场监督管理局(知识产权局)“习近平新时代中国特色社会主义思想蕴含的世界观”}列宁在书中通过对“物理学唯心主义”的批判,确立了物质不依赖于意识而存在的客观地位。

这要求我们必须尊重客观事实。在面对复杂的社会现象或科学问题时,我们不能从主观愿望或抽象的定义出发,篡改客观规律使其符合主观意志,而必须从客观存在的实际情况出发。在进行科研实验时,必须忠实于实验数据,不能为了得到“完美”的实验结论而篡改实验数据,而是面对“不完美”的数据,对实验过程加以反思,分析导致不完美的因素,对实验方法或理论加以修正,使物理理论能够“完美”地解释“不完美”的实验结果。而在修读“大学物理实验”课程时,存在为了获得实验报告更高的分数,篡改实验数据的现象,这是应当反对并加以校正的。

此外,坚持物质第一性意味着要反对一切形式的主观主义和唯意志论。我们不能幻想通过单纯的“意念”或“精神力量”去改变现实,例如美国超级英雄电影中大肆宣扬渲染的“主角光环”或“超能力”\footnote{例如漫威的复仇者联盟系列},而必须通过切实的物质实践活动去改造世界。只有在承认了世界的客观性前提下,我们的主观能动性才能在尊重客观规律的基础上被发挥。

\subsection*{(二)把握认识论规律}
把握认识论规律,核心在于深刻理解绝对真理与相对真理的辩证关系。列宁指出:“绝对真理是由相对真理的总和构成的。”\footnote{列宁. 唯物主义和经验批判主义. 列宁选集(第18卷). 第二章第五节.}这一观点对于我们今天的学习和科研具有极强的指导价值。

首先,我们要树立科学的真理观,反对绝对主义和相对主义两种极端。在学习过程中,我们既不能把现有的理论当作已经僵死的教条,认为它们是终极真理而拒绝发展;与此同时,也不能因为科学理论会随着时代更新的相对真理而否定其客观真理的成分,认为其毫无价值,陷入不可知论的泥潭。例如,牛顿力学虽然被相对论修正,但它在宏观低速领域的真理性依然存在,在我们的日常生活中仍然存在着不可或缺的重要意义,仍然是我们大学物理学习中最为基础重要的部分之一。我们应当认识到,每一个科学结论都是人类真理大海中的一滴相对真理,它包含着绝对真理的部分,但又为历史条件所局限,人类对于知识的探索,正是从相对真理汇集为绝对真理的过程。

同时,我们还应当坚持实践是检验真理的唯一标准。认识不是一次完成的,而是一个“实践、认识、再实践、再认识”\footnote{马克思主义原理教材}的无限循环过程。作为新时代的大学生,我们不仅要在课堂上学习教材中的知识,更要注重社会实践和科学实验。只有在实践中,我们才能检验知识的真伪,发现新的问题,从而推动认识的不断深化,“纸上得来终觉浅,绝知此事要躬行”。只有勇于实践、敢于创新的精神,我们才能真正熟练地掌握并能够运用知识。

\subsection*{(三)正确看待自然科学变革中的“危机”}
列宁在书中深刻分析了当时的“物理学危机”,并一针见血地指出这其实是“形而上学唯物主义的危机”,而不是“唯物主义的崩溃”。这一论断具有深远的指导意义,当今时代科技发展日新月异,自然科学领域同样面临着前所未有的变革和挑战,列宁的唯物主义的对待科学变革而产生的危机的态度,给了我们重要的启示。

当今世界,人工智能技术的飞速发展,特别是以ChatGPT为代表的生成式AI的出现,在人群中引发了新一轮的“认知危机”。面对机器通过算法展现出的高智能高效率,人们或许难免产生“世界也不过是一个虚拟程序”的怀疑。这正与列宁当年面对的“原子非物质化”论调不谋而合。然而,根据列宁在书中阐述的理论,这并非唯物主义的崩溃,而是形而上学唯物主义的破产。AI的智能表现并没有脱离物质载体,它依然是建立在硅基芯片、电路理论和计算机科学算法之上的客观实在。这恰恰是物质运动的形式丰富性的有力佐证\footnote{坦白地说,我不知道如何严谨地表述这个句子}:物质不仅可以演化出人脑这样高度复杂且具有巨大智能的有机体,也能通过人的实践创造出具有高级信息处理能力的智能机器,而这背后的支撑,是诸如信息论、优化理论、统计力学与随机过程等重要基础科学的理论的支撑。科学的每一次突破,并没有对唯物主义的实质产生冲击,相反,其在某些方面上为辩证唯物主义提供了更为有力的佐证。

此外,自然科学与哲学之间也存在着紧密联系。列宁曾告诫:“任何自然科学,任何唯物主义,如果没有充分可靠的哲学论据,是无法对资产阶级思想的侵袭和资产阶级世界观的复辟坚持斗争的。为了坚持这个斗争,为了把它进行到底并取得完全胜利,自然科学家就应该做一个现代的唯物主义者,做一个以马克思为代表的唯物主义的自觉信徒,也就是说应该做一个辩证唯物主义者。”\footnote{列宁. 论战斗唯物主义的意义. 列宁选集(第43卷).}作为光电信息科学与工程专业的学生,我们在深入钻研专业知识、探索光电信息奥秘的同时,更有必要自觉提高哲学素养。只有掌握了辩证唯物主义这一科学的世界观和方法论,我们才能在探索自然奥秘的过程中不迷失方向,透过复杂的现象把握事物的本质,避免滑向唯心主义或形而上学的深渊,真正成为一名既有科学精神又深刻理解马克思主义基本原理的新时代科技人才。

% % 参考文献
% \newpage
% {\heiti\zihao{3}\textbf{参考文献}}

% \begin{thebibliography}{99}
%     \bibitem{reference1} 列宁. 唯物主义和经验批判主义[M]. 北京:人民出版社,2014.
%     \bibitem{reference2} 中共中央马克思恩格斯列宁斯大林著作编译局. 列宁选集(第二卷)[M]. 北京:人民出版社,2012.
%     \bibitem{reference3} 曹浩瀚. 列宁《唯物主义和经验批判主义》的当代价值[J]. 马克思主义研究,2019(05):25-34.
%     \bibitem{reference4} 孙正聿. 辩证唯物主义的“世界图景”——重读《唯物主义和经验批判主义》[J]. 哲学研究,2009(04):3-10.
%     \bibitem{reference5} 顾海良. 马克思主义发展史[M]. 北京:中国人民大学出版社,2009.
% \end{thebibliography}

\end{document}
